\documentclass{article}
\usepackage{lmodern}

% Use pdf version 1.6 or higher
\pdfminorversion=6

% Math font
\usepackage{amsmath, amsthm, amssymb}
\DeclareMathOperator*{\argmax}{argmax}
\usepackage{bm}
\setcounter{MaxMatrixCols}{20}

% Configure graphic
\usepackage{graphicx}
\usepackage{subcaption}
\usepackage{float}
\graphicspath{{figures/}}

% Change text's size
\usepackage{scrextend}
\changefontsizes{12}

% Remove identation at beginning of paragraph
%\usepackage{indentfirst}
\usepackage{parskip}
\setlength{\parindent}{0pt}

% Change line spacing
\renewcommand{\baselinestretch}{1.5}

% Format document's shape
\usepackage[left=1.0cm, right=1.0cm, top=2.0cm, bottom=2.0cm]{geometry}

% More beautiful quotation marks
\usepackage{csquotes}                

% Color tex
\usepackage{color}

% Configure reference
% Note in TeXstudio:
% Options > Configure TeXstudio > Build > Defaut Bibbliography Tool
% Change "BibTeX" to "Biber"
% Bib file must incule extension
\usepackage[style=nature, backend=biber]{biblatex}
\addbibresource{reference.bib}

% Creat bookmark for PDF file
\usepackage{hyperref}

\usepackage{booktabs}
\usepackage{bigstrut}

% Set the depth of table of contents
\setcounter{tocdepth}{5}

% Extend section numbering
\setcounter{secnumdepth}{3}

% Write pseudo-code
\usepackage{algorithm}
\usepackage[noend]{algpseudocode}

\usepackage{multirow}
\usepackage{longtable}
\usepackage{makecell}
\usepackage{diagbox}
\usepackage{appendix}

\DeclareRobustCommand{\&}{%
	\ifdim\fontdimen1\font>0pt
	\textsl{\symbol{`\&}}%
	\else
	\symbol{`\&}%
	\fi
}

% Some customized comments to make life easier
\newcommand{\blue}[1]{\textcolor{blue}{#1}}
\newcommand{\red}[1]{\textcolor{red}{#1}}


% Configure titile
\title{\vspace{-40pt}\Large RLU estimation}
\author{\normalsize Thai Nguyen}
\date{}

\begin{document}
\maketitle

\section{Introduction}
\subsection{Problem formulation}


\subsection{C-MAPSS dataset}
This dataset was generated with the C-MAPSS simulator. C-MAPSS stands for \red{\enquote{Commercial Modular Aero-Propulsion System Simulation}} and it is a tool for the simulation of realistic large commercial turbofan engine data. The C-MAPSS dataset contains 4 sub-datasets generated under different operating and fault conditions which are further divided into training and test sets. A summarization of this dataset is given as belows:

\begin{table}[H]
	\centering
	\begin{tabular}{lcccc}
		\hline
		Dataset                         & FD001 & FD002 & FD003 & FD004 \\ \hline \hline
		Number of training trajectories & 100   & 260   & 100   & 249   \\ \hline
		Number of test trajectories     & 100   & 259   & 100   & 248   \\ \hline
		Number of operating conditions  &       &       &       &       \\ \hline
		Number of fault modes           &       &       &       &       \\ \hline
	\end{tabular}
	\caption{C-MAPSS dataset } 
\end{table} 

The data are provided as a zip-compressed \blue{text} file with \blue{26 columns} of numbers, \blue{separated by spaces}. Each row in the data is a snapshot of data taken during a \red{single operating time cycle}, which includes 26 columns:
\begin{itemize}
	\item The first column represents the \red{engine ID}.
	\item The second column represents the \red{current operational cycle number}.
	\item Columns from 3 to 5 are the \red{three operational settings} that have substantial effects on engine performance.
	\item Columns from 6 to 26 represent the \red{21 sensor values}.
\end{itemize}
\subsection{Objective}
The goal is to \red{predict the number of remaining operational cycles before failure in the test set}, i.e., the number of operational cycles after the last cycle that the engine will continue to operate.

\section{Data processing}
\subsection{Piece-wise linear RUL target function}
\subsection{Data normalization}
The goal of normalization is to transform features to be on a similar scale which can improves the performance and training stability of the model. It is often used when data futures are on \red{drastically difference scales}.

\subsubsection{Max-min normalization}
\subsubsection{Clipping}
\subsubsection{Log scaling}
\subsubsection{Z-score}
\printbibliography

\end{document}














