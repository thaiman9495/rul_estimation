\documentclass{article}
\usepackage{lmodern}

% Use pdf version 1.6 or higher
\pdfminorversion=6

% Math font
\usepackage{amsmath, amsthm, amssymb}
\DeclareMathOperator*{\argmax}{argmax}
\usepackage{bm}
\setcounter{MaxMatrixCols}{20}

% Configure graphic
\usepackage{graphicx}
\usepackage{subcaption}
\usepackage{float}
\graphicspath{{figures/}}

% Change text's size
\usepackage{scrextend}
\changefontsizes{12}

% Remove identation at beginning of paragraph
%\usepackage{indentfirst}
\usepackage{parskip}
\setlength{\parindent}{0pt}

% Change line spacing
\renewcommand{\baselinestretch}{1.5}

% Format document's shape
\usepackage[left=1.0cm, right=1.0cm, top=2.0cm, bottom=2.0cm]{geometry}

% More beautiful quotation marks
\usepackage{csquotes}                

% Color tex
\usepackage{color}

% Configure reference
% Note in TeXstudio:
% Options > Configure TeXstudio > Build > Defaut Bibbliography Tool
% Change "BibTeX" to "Biber"
% Bib file must incule extension
\usepackage[style=nature, backend=biber]{biblatex}
\addbibresource{reference.bib}

% Creat bookmark for PDF file
\usepackage{hyperref}

\usepackage{booktabs}
\usepackage{bigstrut}

% Set the depth of table of contents
\setcounter{tocdepth}{5}

% Extend section numbering
\setcounter{secnumdepth}{3}

% Write pseudo-code
\usepackage{algorithm}
\usepackage[noend]{algpseudocode}

\usepackage{multirow}
\usepackage{longtable}
\usepackage{makecell}
\usepackage{diagbox}
\usepackage{appendix}

\DeclareRobustCommand{\&}{%
	\ifdim\fontdimen1\font>0pt
	\textsl{\symbol{`\&}}%
	\else
	\symbol{`\&}%
	\fi
}

% Some customized comments to make life easier
\newcommand{\blue}[1]{\textcolor{blue}{#1}}
\newcommand{\red}[1]{\textcolor{red}{#1}}


% Configure titile
\title{\vspace{-40pt}\Large RLU estimation}
\author{\normalsize Thai Nguyen}

\begin{document}
\maketitle

\section{Why RUL estimation?}
\section{Mathematical formulation for RUL estimation problem}
Suppose that we are provided with run-to-failure historical data of multiple machines of the same type. The historical data is supposed to cover a representative set of the considering machine type. The recored data of each machine ends when it reach a failure condition threshold. The objective is to learn a RUL model to predict the remaining lifetime of the considering machine type.

\subsection{Training data}
The training data is a set of the operation history of $N$ machines of the same type denoted as $\mathcal{X} = \left\lbrace X^n \mid n=1, ..., N \right\rbrace$. There are $M$ measurements that quantify their health behavior during their operations (e.g. sensors that are installed on these machine to monitor their conditions). The data from the $n^{th}$ machine throughout its useful lifetime produces a multivariate time series $X^n \in \mathbb{R}^{T^n \times M}$ in which $T^n$ denotes the total number of time steps of machine $n$ throughout is lifetime (in other words, $T^n$ is the failure time of component $n$). We use the notation $X_t^n \in \mathbb{R}^M$ to denote the $t^{th}$ timestamp of $X^n$ where $t \in \left\lbrace 1, ..., T^n \right\rbrace$. Indeed, $X_t^n$ is a vector of $M$ sensor values.

\subsection{Testing data}
The test set consists of historical data of $K$ machines of the same type used in the training data denoted as $\mathcal{Z}= \left\lbrace Z^k \mid k=1, ..., K \right\rbrace$ in which $Z^k$ is a time series of $k^{th}$ machine. The notation $Z_t^k \in \mathbb{R}^M$ is used to denote the $k^{th}$ timestamp of $Z^k$ where $t \in \left\lbrace 1, ..., L^k \right\rbrace$ where $L^k$ is the total number of time steps related to machine $k$ in the test set. Obviously, the test set will not consist of all time steps up to the failure point, i.e.., $L^k$ is generally smaller than the the failure time of component $k$ denoted as $\bar{L}^k$.

We focus on estimating RUL of component $k$, $\bar{L}^k - L^k$, given the data from time step 1 to $L^k$. Note that $\bar{L}^k - L^k$ is also provided in the test set.

\section{CMAPSS dataset}
\section{Data preparation}
\subsection{RUL target}
We can generate the RUL for very time steps in a training trajectory $X^n$ based on $T^n$. In the literature, there are two common models for generating RUL given the failure time, namely, linear and piece-wise linear model degradation model. These two models are mathematically presented in the following.

\paragraph{Linear model degradation model} This kind of RUL model is very obvious considering the fact that we have the the failure point of each training trajectory $(T^n)$. The RUL of machine $n$ at time step $t$ in the training set, $R_t^n$, is calculated as belows:
\begin{equation}
	R_t^n = T^n - t
\end{equation}

\paragraph{Piece-wise linear degradation model}  Since the degradation of a machine will generally not be noticeable unit it has been operating for some period of time. Therefore, it is probably reasonable to estimate RUL of a machine until it begins to degrade. For this reason, it seem to be ok to estimate the RUL when the machine is new as constant. As a result, the piece-wise linear degradation model is proposed to set an upper limit on the RUL target as belows:
\begin{equation}
	 R_t^n = 
	\begin{cases}
		R_t ^n = R^{max}  & \text{if } t \le T^{max} \\ 
		R_t^n = T^n - t   & \text{otherwise} 
	\end{cases}
\end{equation}

\subsection{Data normalization}



\printbibliography

\end{document}














